\documentclass{article}
\usepackage[utf8]{inputenc}
\usepackage{graphicx}

\title{ Wiki Distribué } 
\author{ Erick Lavoie \\ Frédéric van der Essen}



\begin{document}
	\maketitle
	\section{Mode d'emploi} 
	\begin{figure}[h]
		\centering
		\includegraphics[scale=0.5]{connection.png}
		\caption{L'application en phase de connection}
	\end{figure}
	\begin{figure}[h]
		\centering
		\includegraphics[scale=0.5]{connected.png}
		\caption{L'application en phase de navigation}
	\end{figure}
	
	\subsection{Se connecter au réseau}
	La première étape lorsque l'on ouvre le programme est de se connecter
	à un réseau pair à pair. Notre implémentation permet à l'utilisateur de
	choisir deux options. 
	
	\emph{Host Server} initialise un nouveau réseau
	pair à pair avec un nombre suffisants de noeuds pour que celui-ci puisse
	fonctionner correctement. Ensuite, l'application s'y connecte et renvoie
	à l'utilisateur un ticket oz pour que d'autres applications puissent s'y
	connecter. Dans ce cas, l'ensemble du réseau pair à pair tourne dans
	un seul processus.
	
	\emph{Connect to Server} se connecte à un réseau pair à pair existant.
	Pour ce faire, l'utilisateur doit disposer d'un ticket oz. Si un serveur
	tourne déjà sous le même utilisateur Unix, alors l'application ira
	automatiquement chercher le ticket dans un fichier se trouvant dans
	le répertoire utilisateur. Sinon, l'utilisateur devra trouver son ticket
	oz par mail ou sur un site web. 
	
	\subsection{Navigation des pages web}
	Une fois connecté à un réseau, l'utilisateur entre sur la page par défaut
	\emph{home}. L'utilisateur peut se rendre sur n'importe quelle page en
	entrant son url dans la barre du haut. Si cette page n'existe pas, 
	une nouvelle sera automatiquement créée. l'url n'a pas de format spécifique
	et peut être n'importe quelle string. 
	
	L'utilisateur peut aussi rafraichir une page ce qui lui permet d'accéder
	à la dernière version.
	
	\subsection{Édition des pages web}
	Lorsque l'utilisateur est sur une page, il peut la modifier à loisir.
	Pour cela il suffit de l'éditer. Après chaque changement dans un paragraphe,
	l'utilisateur doit les sauvegarder. Une fois tous ses changements
	effectués et sauvegardés il peut les envoyer sur le réseau. En cas
	de conflit dans un paragraphe, les changements ne sont pas effectués,
	l'utilisateur est notifié, et la page est rafraichie à sa version la plus
	récente.  Actuellement, il n'est possible d'ajouter ou de supprimer
	qu'un seul paragraphe par sauvegarde.  Cependant, il est possible de
	soumettre le résultat de plus d'une sauvegarde sur le réseau.
	
	\section{Validation des requis}
	Les deux tests présentés ci-après montrent la gestion des différentes modifications,
	lors d'un conflit et lors d'une modification concurrente réussie.
	
	\subsection{Gestion d'un conflit d'édition}
	Le plus simple cas pour réaliser un conflit d'édition est de suivre les étapes suivantes:
	\begin{itemize}
		\item Démarrer un premier client
		\item Démarrer l'anneau de noeuds ''server'' sur un des deux clients.
		\item Démarrer un deuxième client et se connecter au premier.  Par défaut, le ticket
			du premier sera déjà dans la barre d'adresse.
		\item Sur le premier client, modifier la ligne de texte, sauvegarder et soumettre les
			changements.
		\item Sur le deuxième, modifier également le texte, sauvegarder et soumettre le texte. 
			Le deuxième client notifiera que la modification ne peut être effectuée et chargera la dernière
			page valide sur le serveur.	
	\end{itemize}
	
	\subsection{Gestion de modifications concurrentes indépendantes}
	Le plus simple cas pour constater l'intégration de modifications concurrentes est de suivre les étapes suivantes:
	\begin{itemize}
		\item Démarrer un premier client
		\item Démarrer l'anneau de noeuds ''server'' sur un des deux clients.
		\item Démarrer un deuxième client et se connecter au premier.  Par défaut, le ticket
			du premier sera déjà dans la barre d'adresse.
		\item Sur le premier client, ajouter un paragraphe en dessous du premier (actuellement un 
			seul paragraphe peut être ajouté à la fois entre les différentes sauvegarde).
			Sauvegarder et soumettre les modifications.
		\item Sur le deuxième, rafraîchir la page modifiée par le premier client.
		\item Sur le premier client, modifier le premier paragraphe, sauvegarder et soumettre.
		\item Sur le deuxième, modifier le deuxième paragraphe, sauvegarder et soumettre. Les 
			changements effectués sur le premier client vont apparaître.
		\item Rafraîchir la page sur le premier client pour voir apparaître les changements du 
			deuxième client.	
	\end{itemize}
	
	\section{Architecture}
	Le système actuel dédie un noeud par client qui se joint à l'anneau.  Étant donné les différents 
	problèmes de connexion rencontrés, nous nous en sommes tenus à la version simple de l'anneau, 
	sans utiliser la possibilité de ''relaxer'' l'anneau avec des branches.  Pour assurer la majorité lors des
	différentes opérations distribuées, nécessaire pour réaliser les ''commit'' avec l'algorithme Paxos,
	un anneau initial composé de 5 noeuds est initialisé lorsque la fonction ''host server'' est utilisée.
	L'anneau tourne alors dans le même processus Oz que le client.  Tous les clients suivants qui
	se connectent auront un noeud dans leur processus qui se connectera au processus initial.  Cette topologie
	a été utlisée pour la simple et unique raison qu'elle minimisait les risques d'erreurs de connexion au moment 
	où un noeud dans un autre processus se joint à l'anneau, puisque plusieurs erreurs ont été 
	rencontrées au courant du développement.
	
	
	\section{Design}
	Le design du système s'est articulé autour de deux principaux points: la gestion des transactions et le développement
	d'un type de données.
	
	\subsection{Transactions}
	Deux transactions ont été identifiées, \textit{Refresh} et \textit{Submit}.  
	
	La première permet de récupérer la page associée à une URL donnée.  Dans le cas, où aucune page 
	n'est associée à l'URL, alors une nouvelle page est crée.  Ce comportement
	correspond à celui de wikipedia, qui crée une nouvelle page en mode édition lorsqu'un lien qui pointe vers une page
	inexistante est suivi.  Cette transaction réussit donc toujours et retourne toujours une page valide.
	
	La deuxième permet de mettre à jour la page modifiée.  Pour ce faire, la transaction récupère la page
	courante et tente de fusionner les changements effectués sur la page.  Si cette fusion réussit alors
	la transaction réussit sinon elle est annulée.  Voir la section suivante pour le type de donnée page.
	
	Les transations sont dans le functor Transactions.oz sous lib/ dans le code source.
	
	\subsection{Type de donnée Page}
	Pour pouvoir effectuer correctement la mise à jour des pages, un type de donnée symbolique page a été réalisé,
	supportant les ajouts, les retraits et la mise à jour de paragraphes.  Chaque paragraphe est versionné de sorte,
	qu'il est possible de discriminer l'ordre temporel des changements.  De plus, il y a séparation entre les paragraphes
	et leur position dans la page ce qui permet de découpler leur position de leur contenu et permettre une fusion intelligente.
	 La fonction de fusion des pages en version symbolique est robuste et permet de suivre les modifications à 
	 des paragraphes mêmes si ceux si ont été déplacés ou que des paragraphes ont été ajoutés ou supprimés.  
	 
	 Cependant, comme l'utilisateur ne modifie pas directement cette version symbolique, des fonctions permettent de passer
	d'une version symbolique à une version textuelle (String) et vice versa.  L'algorithme actuelle permet de détecter
	des modifications sur plusieurs paragraphes en même temps mais ne supporte l'ajout ou la suppression que d'un
	paragraphe à la fois.  Étant donné le temps limité pour arriver à une solution transparente pour l'utilisateur,
	il a été choisi de demander à l'utilisateur de sauvegarder en version symbolique manuellement la version qu'il est 
	en train d'éditer.  Il est possible de procéder à plusieurs sauvegardes avant de fusionner avec la version courante
	car l'algorithme de fusion est assez robuste.

	\newpage	
	Voici la page par défaut affichée ainsi que l'explication des différents champs de la page.

	\begin{verbatim}
	page(1:1  // Position dans la page:ID du paragraphe
	           content:content(
	                             1:paragraph(content:''To be filled.'' 
	                                                      version:1) 
	                              count:1)  // Nb de paragraphes,
	                               	        // pour generer des ID uniques
	           highestposition:1 // Nb de paragraphe dans la page
	           version:1) 
	\end{verbatim}	
	
	Les pages et les opérations associées sont dans le functor Page.oz sous lib/ dans le code source.
	
	
	\subsection{Limitations et Améliorations potentielles}
	Notre application actuelle remplie les exigences telle que présentées dans le document de requis
	mais ces exigences semblent clairement insuffisantes pour une utilisation réelle.  Voici donc une
	explication des limitations rencontrées qui devraient être surpassées.
	\subsubsection{Connectivité} 
		Pour l'instant, pour se connecter, l'utilisateur doit posséder
		un ticket oz vers le processus ayant instancié le réseau.
		Si jamais ce processus n'existe plus il sera impossible pour de
		nouveaux clients de se connecter au réseau. 
		
		Pour résoudre ce problème il faudrait que chaque client puisse
		générer un ticket à partir duquel l'on puisse se connecter.
		
	 \subsubsection{Modifications Concurrentes}
	 	Si l'algorithme de merge du format symbolique des pages est
	 	assez intelligent, l'algorithme qui génère ce format symbolique
	 	ne l'est pas suffisemment. 
	 	
	 	Les conséquences sont qu'il faut enregistrer localement à 
	 	chaque modification de facon manuelle, qu'il ne peut y avoir qu'une modification par
	 	paragraphe, et qu'on ne peut supprimer qu'un seul paragraphe par
	 	transaction. 
	 	
	 	Pour résoudre ces problèmes, il faudrait garder une copie de
	 	la version originale aux cotés de la version modifiée par l'utilisateur.
	 	Cela permettrait de ne pas devoir augmenter les numéros de versions
	 	de paragraphe à chaque modification, et la sauvegarde locale peut
	 	se faire automatiquement, ce qui rendrait l'utilisation de notre
	 	programme plus conviviale. 
	 	
	 	Un autre souci est la gestion des conflits de transaction.
	 	Pour l'instant les changements sont annulés. Il serait plus
	 	intéressant de renvoyer une version contenant les conflits à résoudre
	 	comme le font les Systêmes de contrôle de versions.
		
	\section{Bugs rencontrés}
	Nous n'avons pas eu le temps d'établir une procédure de reproduction pour les deux bugs
	rencontrés ce qui limite l'utilité de la présente section.  Cependant, ceux-ci sont quand même 
	présentés pour que vous puissiez sonder rapidement le bugs rencontrés par chacune des équipes.
	
	Le premier bug est une faute de segmentation rencontrée typiquement lors de la connexion de 
	deux processus oz lors de la prise d'un ticket.  Nous rencontrons ce bug lors du démarrage
	de l'application.  Cependant, à partir du moment où la connexion est établie nous n'avons pas 
	rencontré de faute de segmentation.  Il est suspecté que ce soit un bug de l'implémentation de la version 1.4.
	
	Le deuxième bug est l'impossibilité de joindre un ring sur os x.  Bien que la référence vers l'anneau
	puisse être récupérée correctement à partir du ticket oz, ce qui laisse entendre que la connexion entre 
	plusieurs processus oz est réalisée correctement, l'exécution de join() renvoie systématiquement
	l'exception failedToJoin, même en l'absence de connexion réseau et après avoir désactivé le parefeu.
\end{document}
